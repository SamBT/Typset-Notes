\documentclass[letterpaper]{article}

\usepackage{amsmath}
\usepackage{amssymb}
\usepackage{amsfonts}
\usepackage{dsfont}
\usepackage{fullpage}
\usepackage[colorlinks=true]{hyperref}
\usepackage{cleveref}
\usepackage{physics}
\usepackage{slashed}
\usepackage{fancyhdr}
\usepackage{mathtools}
\usepackage{feynmp-auto}
\usepackage{graphicx}
\usepackage{epstopdf}
\usepackage{tikz}
\usepackage{pgfplots}
\usepackage{tensor}
\usepackage{simplewick}
\usepackage{braket}


\usetikzlibrary{arrows}

\pgfplotsset{width=7cm,compat=1.9}

\newcommand{\VEV}[1]{\bra{0} #1 \ket{0}}
\newcommand{\pd}[1]{\partial_{#1}}
\newcommand{\pu}[1]{\partial^{#1}}

\pagestyle{fancy}
\lhead{Preskill QFT Notes}
\renewcommand{\headrulewidth}{0.4pt}
\setlength\parindent{18pt}

\begin{document}
	\setlength{\headheight}{12pt}
	\setlength{\headsep}{0.2in}

	\section{Quantum Electrodynamics}

	\subsection{The Trouble With Higher Spin}
	Armed with our functional methods, we may now proceed to the quantization of electrodynamics. What is the origin of the technical problems that make functional methods so useful in treating QED?
	\par So far, we have dealt with field theories in which the fields transform as a representation of the Lorentz group that transforms irreducibly under rotations (the \underline{little group} for massive particles):
	\begin{align*}
	\text{(Real) Scalars} \quad &(0,0) \rightarrow \quad \text{Spin 0} \\
	\text{(Weyl) Fermions} \quad &(\tfrac{1}{2},0) \rightarrow \quad \text{Spin } \tfrac{1}{2}
	\end{align*}
	This means that the number of field components is right for the field to create particles that transform irreducibly under the Poincare group when they act on the vacuum. If the transformation law of the field is
	$$U(a,\Lambda)\phi_\alpha(x)U(a,\Lambda)^{-1} = D_{\alpha\beta}\left(\Lambda^{-1}\right)\phi_\beta(\Lambda x + a),$$
	then consider $\tilde{\phi}_\alpha(t = 0, \vec{p} = 0)\ket{0}$; i.e. $\phi$ Fourier transformed in $\vec{x}$ at fixed time. If $R$ is a \underline{rotation} (leaving $\vec{p} = 0$ fixed), then
	$$U(R)\tilde{\phi}_\alpha \ket{0} = D_{\alpha\beta}(R^{-1})\tilde{\phi}_\beta \ket{0}.$$
	This, the states $\tilde{\phi}_\alpha \ket{0}$ transform under the "little group" as the representation $D$ restricted to the $SO(3)$ subgroup of the Lorentz group (this is the little group for massive particles). Under the full Poincare group, the states transform irreducibly if they transform irreducibly under the little group.
	\par But consider, for example, a field that transforms as a Lorentz \underline{4-vector}.
	$$A^\mu (x): \qquad (\tfrac{1}{2},\tfrac{1}{2}) \rightarrow \quad \text{Spin 1 + Spin 0}$$
	The states created by the $A^\mu$ acting on the vacuum are not purely spin 1; $A^\mu$ couples to both spin 0 and spin 1. If we desire to construct a covariant field theory for spin 1 particles, we are burdened with an unwanted spin 0 particle that we will need to remove from the quantum theory somehow.
	\par For spin 2 things are even worse:
	$$h^{\mu\nu}(x): \qquad (1,1) \rightarrow \quad \text{Spin 2 + Spin 1 + Spin 0}$$
	(where $h^{\mu\nu}$ is a symmetric traceless tensor).
	\par We are stuck in a dilemma. A field theory of (for example) spin 1 particles requires either:
	\begin{enumerate}
	\item[a.)] Loss of covariance
	\item[b.)] Unphysical particles
	\end{enumerate}
	\par For \underline{massless} particles things are of course even worse. We want the 4-component field $A_\mu$ to create \underline{two} physical helicity states $\lambda = \pm 1$ (this is a reducible rep, but we know we have to put them together to get a CPT-invariant theory). So in a covariant formulation we will have \underline{two} unphysical degrees of freedom.
	\par Because it is computationally much simpler, we will quantize electrodynamics by introducing \underline{unphysical} degrees of freedom, rather than by breaking Lorentz invariance. Of course, if the theory really makes sense, the two methods should be equivalent. The apparent loss of covariance must be illusory, and not afflict physical quantities. Similarly, the unphysical degrees of freedom must "decouple" from physical processes. (E.g., they are not produced in scattering experiments). Functional methods provide the most efficient way of dealing with such unphysical excitations and showing that they are harmless.
	\par \textbf{Note:} We could attempt to construct a field theory for spin 1 from an antisymmetric tensor field (self-dual):
	$$B_{\mu\nu}: \qquad (1,0) \rightarrow \quad \text{Spin 1}$$
	where $B_{\mu\nu}$ is a self-dual antisymmetric tensor. But such a theory would not be parity invariant. To obtain a parity invariant theory, we would need the representation $(1,0) + (0,1)$ -- so this would be a theory of \underline{two} spin-1 particles rather than one, unless some degrees of freedom are unphysical. (The self dual and anti-self-dual components of the electromagnetic field are $\vec{E} \pm i\vec{B}$ which, of course, are not independent).

	\subsection{The Free Vector Field}
	We will now construct the free field theory for a vector field $A^\mu (x)$. Ultimately we will be interested in a massless vector (the photon), but we will not yet impose masslessness as a condition on the theory. We will consider the theory of a \underline{massive} vector first, and treat the photon as the massless limit eventually.
	\par To begin, we construct the most general Lagrangian quadratic in $A^\mu$ that obeys the usual principles: it is local, Poincare invariant, Hermitian, and depends on no derivatives higher than the first. Possible terms are:
	\begin{align*}
	\text{No derivatives: } \qquad &A_\mu A^\mu \\
	\text{One derivative: } \qquad &\partial_\mu \partial^\mu \quad \text{(total derivative, does not contribute to the action)} \\
	\text{Two derivatives: } \qquad &\partial_\mu A^\nu \partial^\mu A_\nu \\
	& \partial_\mu A^\mu \partial_\nu A^\nu \rightarrow -A^\mu \partial_\mu\partial_\nu A^\nu + \text{ tot. deriv.} \\
	& \partial_\mu A^\nu \partial_\nu A^\mu \rightarrow -A^\nu \partial_\mu\partial_\nu A^\mu + \text{ tot. deriv.}
	\end{align*}
	The last two are not independent, as they are related by integration by parts.
	\par The most general quadratic Lagrangian is
	$$\cal{L} = \pm \frac{1}{2}\left[\partial_\mu A^\nu\partial^\mu A_\nu + a\partial_\mu A^\nu\partial_\nu A^\mu - \mu^2A_\mu A^\mu\right],$$
	where the field has been appropriately rescaled to fix the normalization of the first term, The classical field equation is:
	$$0 = \partial_\mu \frac{\partial\cal{L}}{\partial\partial_\mu A^\nu} - \frac{\partial\cal{L}}{\partial A^\nu} = \partial_\mu\left(\partial^\mu A_\nu + a\pd{\nu}A^\mu + \mu^2A_\nu\right)$$
	or
	$$0 = \pd{\mu}\pu{\mu}A_\nu + a\pd{\mu}\pd{\nu}A^\mu + \mu^2A_\nu$$
	\par What are the plane wave solutions? They have the form
	$$A_\mu (x) = e_\mu e^{-ik\cdot x}$$
	where
	$$0 = -k^2e_\nu - ae\cdot kk_\nu + \mu^2e_\nu$$
	We may distinguish two types of solutions:
	\begin{enumerate}
	\item[1.)] \underline{$e \cdot k = 0$} (Transverse) \\
	For these solutions we have $k^2 = \mu^2$. In the rest frame, $k = (\mu, \vec{0})$, there are three values for $e_\mu$:
	\begin{align*}
	e^{(1)} &= (0,1,0,0) \\
	e^{(2)} &= (0,0,1,0) \\
	e^{(3)} &= (0,0,0,1)
	\end{align*}
	For other frames, we obtain a basis:
	$$e_\mu^{(r)}(k) \qquad r = 1,2,3$$
	by boosting the above basis.
	In any frame, these satisfy
	$$e^{(r)}(k)\cdot e^{(s)}(k) = -\delta^{rs} \quad \text{(invariant inner product)}$$
	since this is true in the rest frame. We also have
	$$\sum_{r} e_\mu^{(r)}e_\nu^{(r)} = -\eta_{\mu\nu} + \frac{k_\mu k_\nu}{\mu^2}$$
	-- this is true in the rest frame, and since it is a covariant relation, is true in any frame.
	\item \underline{$e \propto k$} (longitudinal)
	For these solutions, we have $(1+a)k^2 = \mu^2$ or
	$$k^2 = \mu_L^2 \equiv \frac{\mu^2}{1+a}$$
	In the rest frame $k = (\mu_L, \vec{0})$. We see that
	$$e^{(L)} = (1,0,0,0) = \frac{k}{\mu_L} \quad \text{(A spin 0 rep. of the little group)}$$
	In an arbitrary frame, this evidently boosts to
	$$e^{(L)} = \frac{1}{\mu_L}k, \quad e^{(L)}\cdot e^{(r)} = 0, \quad e^{(L)^2} = 1$$
	$$e_\mu^{(L)}e_\nu^{(L)} = \frac{k_\mu k_\nu}{\mu_L^2}$$
	\end{enumerate}
	As we anticipated, we see that the plane wave solutions correspond to a particle of spin 1 and mass $\mu$, created (in the rest frame) by $A^i$ acting on the vacuum; and a particle of spin 0 and mass $\mu_L$, created by $A^0$ acting on the vacuum (the masses can be different because they correspond to distinct irreps of the little group). But we are free to choose $a = -1$. This removes the spin zero longitudinal mode from the theory (pushes its mass to infinity).
	\par If we have removed one of the plane wave solutions from the theory, this must correspond to a \underline{constraint} on the $A^\mu$ that removes one of the classical degrees of freedom from the theory. To see what the constraint is, consider again the classical equation of motion.
	\par With $a = -1$ the classical equation of motion is
	 $$0 = \pu{\mu}\left(\pd{\mu}A_\nu - \pd{\nu}A_\nu)\right) + \mu^2A_\nu$$
	 or
	 $$ 0 = \pu{\mu}F_{\mu\nu} + \mu^2A_\nu$$
	 where $F_{\mu\nu} = \pd{\mu}A_\nu - \pd{\nu}A_\nu$ --  an antisymmetric tensor. But since $F_{\mu\nu}$ is antisymmetric,
	 $$0 = \pu{\mu}\pu{\nu}F_{\mu\nu} = -\mu^2\pu{\nu}A_\nu$$
	 Thus, $\pu{\nu}A_\nu = 0$ (if $\mu^2 \neq 0$) is required for \underline{consistency} (and tells us that plane wave solutions must obey $e \cdot k = 0$). If we plug this condition back into the equations of motion we have
	 $$(\pu{\mu}\pd{\mu} + \mu^2)A^\nu = 0, \qquad \pd{\mu}A^\mu = 0.$$
	 Each component of $A^\nu$ obeys the Klein-Gordon equation for a particle of mass $\mu$. But the field components are \underline{not} independent.

\subsection{Canonical Formulation}
We can understand better the nature of the constraints, and identify easily the physical degrees of freedom, in the canonical formulation of the theory. In the canonical (Hamiltonian) formulation, we have a set of $p$'s and $q$'s, and if the initial values of all independent $p$'s and $q$'s are specified, the subsequent trajectory is completely determined. If the system has $n$ independent degrees of freedom (in the Lagrangian sense -- $n$ $q$'s) then $2n$ initial data are needed and the dimension of phase space ($p$'s and $q$'s) is $2n$.
\par If we wish to canonically quantize, we must, of course, impose canonical commutation relations on a complete set of independent $p$'s and $q$'s.
\par Consider again the Lagrangian
\begin{align*}
\cal{L} &= \pm\left[\frac{1}{4}F_{\mu\nu}F^{\mu\nu} - \frac{1}{2}\mu^2A_\mu A^\mu\right] \\
&= \pm\left[\frac{1}{2}F_{0i}F^{0i} + \frac{1}{4}F_{ij}F^{ij} - \frac{1}{2}\mu^2A_iA^i - \frac{1}{2}A_0A^0 \right]
\end{align*}
The conjugate momenta to $A_i$ and $A_0$ are
$$\frac{\partial\cal{L}}{\partial\pd{0}A_i} = \pm F^{0i}, \quad \frac{\partial\cal{L}}{\partial\pd{0}A_0} = 0$$
$A_0$ has no conjugate momentum. What is going on? Since the derivatives of $A_0$ do not appear in $\cal{L}$, $A_0$ is not really a \underline{dynamical} variable; rather, it is \underline{constrained}. When the other dynamical variables are specified at a given time, $A_0$ is \underline{determined}. To see this more clearly, construct $\cal{H}$ by Legendre transforming with respect to the three variables $A_i$.

\end{document}
